\documentclass[11pt]{article}

    \usepackage[breakable]{tcolorbox}
    \usepackage{parskip} % Stop auto-indenting (to mimic markdown behaviour)
    
    \usepackage{iftex}
    \ifPDFTeX
    	\usepackage[T1]{fontenc}
    	\usepackage{mathpazo}
    \else
    	\usepackage{fontspec}
    \fi

    % Basic figure setup, for now with no caption control since it's done
    % automatically by Pandoc (which extracts ![](path) syntax from Markdown).
    \usepackage{graphicx}
    % Maintain compatibility with old templates. Remove in nbconvert 6.0
    \let\Oldincludegraphics\includegraphics
    % Ensure that by default, figures have no caption (until we provide a
    % proper Figure object with a Caption API and a way to capture that
    % in the conversion process - todo).
    \usepackage{caption}
    \DeclareCaptionFormat{nocaption}{}
    \captionsetup{format=nocaption,aboveskip=0pt,belowskip=0pt}

    \usepackage[Export]{adjustbox} % Used to constrain images to a maximum size
    \adjustboxset{max size={0.9\linewidth}{0.9\paperheight}}
    \usepackage{float}
    \floatplacement{figure}{H} % forces figures to be placed at the correct location
    \usepackage{xcolor} % Allow colors to be defined
    \usepackage{enumerate} % Needed for markdown enumerations to work
    \usepackage{geometry} % Used to adjust the document margins
    \usepackage{amsmath} % Equations
    \usepackage{amssymb} % Equations
    \usepackage{textcomp} % defines textquotesingle
    % Hack from http://tex.stackexchange.com/a/47451/13684:
    \AtBeginDocument{%
        \def\PYZsq{\textquotesingle}% Upright quotes in Pygmentized code
    }
    \usepackage{upquote} % Upright quotes for verbatim code
    \usepackage{eurosym} % defines \euro
    \usepackage[mathletters]{ucs} % Extended unicode (utf-8) support
    \usepackage{fancyvrb} % verbatim replacement that allows latex
    \usepackage{grffile} % extends the file name processing of package graphics 
                         % to support a larger range
    \makeatletter % fix for grffile with XeLaTeX
    \def\Gread@@xetex#1{%
      \IfFileExists{"\Gin@base".bb}%
      {\Gread@eps{\Gin@base.bb}}%
      {\Gread@@xetex@aux#1}%
    }
    \makeatother

    % The hyperref package gives us a pdf with properly built
    % internal navigation ('pdf bookmarks' for the table of contents,
    % internal cross-reference links, web links for URLs, etc.)
    \usepackage{hyperref}
    % The default LaTeX title has an obnoxious amount of whitespace. By default,
    % titling removes some of it. It also provides customization options.
    \usepackage{titling}
    \usepackage{longtable} % longtable support required by pandoc >1.10
    \usepackage{booktabs}  % table support for pandoc > 1.12.2
    \usepackage[inline]{enumitem} % IRkernel/repr support (it uses the enumerate* environment)
    \usepackage[normalem]{ulem} % ulem is needed to support strikethroughs (\sout)
                                % normalem makes italics be italics, not underlines
    \usepackage{mathrsfs}
    

    
    % Colors for the hyperref package
    \definecolor{urlcolor}{rgb}{0,.145,.698}
    \definecolor{linkcolor}{rgb}{.71,0.21,0.01}
    \definecolor{citecolor}{rgb}{.12,.54,.11}

    % ANSI colors
    \definecolor{ansi-black}{HTML}{3E424D}
    \definecolor{ansi-black-intense}{HTML}{282C36}
    \definecolor{ansi-red}{HTML}{E75C58}
    \definecolor{ansi-red-intense}{HTML}{B22B31}
    \definecolor{ansi-green}{HTML}{00A250}
    \definecolor{ansi-green-intense}{HTML}{007427}
    \definecolor{ansi-yellow}{HTML}{DDB62B}
    \definecolor{ansi-yellow-intense}{HTML}{B27D12}
    \definecolor{ansi-blue}{HTML}{208FFB}
    \definecolor{ansi-blue-intense}{HTML}{0065CA}
    \definecolor{ansi-magenta}{HTML}{D160C4}
    \definecolor{ansi-magenta-intense}{HTML}{A03196}
    \definecolor{ansi-cyan}{HTML}{60C6C8}
    \definecolor{ansi-cyan-intense}{HTML}{258F8F}
    \definecolor{ansi-white}{HTML}{C5C1B4}
    \definecolor{ansi-white-intense}{HTML}{A1A6B2}
    \definecolor{ansi-default-inverse-fg}{HTML}{FFFFFF}
    \definecolor{ansi-default-inverse-bg}{HTML}{000000}

    % commands and environments needed by pandoc snippets
    % extracted from the output of `pandoc -s`
    \providecommand{\tightlist}{%
      \setlength{\itemsep}{0pt}\setlength{\parskip}{0pt}}
    \DefineVerbatimEnvironment{Highlighting}{Verbatim}{commandchars=\\\{\}}
    % Add ',fontsize=\small' for more characters per line
    \newenvironment{Shaded}{}{}
    \newcommand{\KeywordTok}[1]{\textcolor[rgb]{0.00,0.44,0.13}{\textbf{{#1}}}}
    \newcommand{\DataTypeTok}[1]{\textcolor[rgb]{0.56,0.13,0.00}{{#1}}}
    \newcommand{\DecValTok}[1]{\textcolor[rgb]{0.25,0.63,0.44}{{#1}}}
    \newcommand{\BaseNTok}[1]{\textcolor[rgb]{0.25,0.63,0.44}{{#1}}}
    \newcommand{\FloatTok}[1]{\textcolor[rgb]{0.25,0.63,0.44}{{#1}}}
    \newcommand{\CharTok}[1]{\textcolor[rgb]{0.25,0.44,0.63}{{#1}}}
    \newcommand{\StringTok}[1]{\textcolor[rgb]{0.25,0.44,0.63}{{#1}}}
    \newcommand{\CommentTok}[1]{\textcolor[rgb]{0.38,0.63,0.69}{\textit{{#1}}}}
    \newcommand{\OtherTok}[1]{\textcolor[rgb]{0.00,0.44,0.13}{{#1}}}
    \newcommand{\AlertTok}[1]{\textcolor[rgb]{1.00,0.00,0.00}{\textbf{{#1}}}}
    \newcommand{\FunctionTok}[1]{\textcolor[rgb]{0.02,0.16,0.49}{{#1}}}
    \newcommand{\RegionMarkerTok}[1]{{#1}}
    \newcommand{\ErrorTok}[1]{\textcolor[rgb]{1.00,0.00,0.00}{\textbf{{#1}}}}
    \newcommand{\NormalTok}[1]{{#1}}
    
    % Additional commands for more recent versions of Pandoc
    \newcommand{\ConstantTok}[1]{\textcolor[rgb]{0.53,0.00,0.00}{{#1}}}
    \newcommand{\SpecialCharTok}[1]{\textcolor[rgb]{0.25,0.44,0.63}{{#1}}}
    \newcommand{\VerbatimStringTok}[1]{\textcolor[rgb]{0.25,0.44,0.63}{{#1}}}
    \newcommand{\SpecialStringTok}[1]{\textcolor[rgb]{0.73,0.40,0.53}{{#1}}}
    \newcommand{\ImportTok}[1]{{#1}}
    \newcommand{\DocumentationTok}[1]{\textcolor[rgb]{0.73,0.13,0.13}{\textit{{#1}}}}
    \newcommand{\AnnotationTok}[1]{\textcolor[rgb]{0.38,0.63,0.69}{\textbf{\textit{{#1}}}}}
    \newcommand{\CommentVarTok}[1]{\textcolor[rgb]{0.38,0.63,0.69}{\textbf{\textit{{#1}}}}}
    \newcommand{\VariableTok}[1]{\textcolor[rgb]{0.10,0.09,0.49}{{#1}}}
    \newcommand{\ControlFlowTok}[1]{\textcolor[rgb]{0.00,0.44,0.13}{\textbf{{#1}}}}
    \newcommand{\OperatorTok}[1]{\textcolor[rgb]{0.40,0.40,0.40}{{#1}}}
    \newcommand{\BuiltInTok}[1]{{#1}}
    \newcommand{\ExtensionTok}[1]{{#1}}
    \newcommand{\PreprocessorTok}[1]{\textcolor[rgb]{0.74,0.48,0.00}{{#1}}}
    \newcommand{\AttributeTok}[1]{\textcolor[rgb]{0.49,0.56,0.16}{{#1}}}
    \newcommand{\InformationTok}[1]{\textcolor[rgb]{0.38,0.63,0.69}{\textbf{\textit{{#1}}}}}
    \newcommand{\WarningTok}[1]{\textcolor[rgb]{0.38,0.63,0.69}{\textbf{\textit{{#1}}}}}
    
    
    % Define a nice break command that doesn't care if a line doesn't already
    % exist.
    \def\br{\hspace*{\fill} \\* }
    % Math Jax compatibility definitions
    \def\gt{>}
    \def\lt{<}
    \let\Oldtex\TeX
    \let\Oldlatex\LaTeX
    \renewcommand{\TeX}{\textrm{\Oldtex}}
    \renewcommand{\LaTeX}{\textrm{\Oldlatex}}
    % Document parameters
    % Document title
    \title{Python\_Oops\_concepts\_revision}
    
    
    
    
    
% Pygments definitions
\makeatletter
\def\PY@reset{\let\PY@it=\relax \let\PY@bf=\relax%
    \let\PY@ul=\relax \let\PY@tc=\relax%
    \let\PY@bc=\relax \let\PY@ff=\relax}
\def\PY@tok#1{\csname PY@tok@#1\endcsname}
\def\PY@toks#1+{\ifx\relax#1\empty\else%
    \PY@tok{#1}\expandafter\PY@toks\fi}
\def\PY@do#1{\PY@bc{\PY@tc{\PY@ul{%
    \PY@it{\PY@bf{\PY@ff{#1}}}}}}}
\def\PY#1#2{\PY@reset\PY@toks#1+\relax+\PY@do{#2}}

\@namedef{PY@tok@w}{\def\PY@tc##1{\textcolor[rgb]{0.73,0.73,0.73}{##1}}}
\@namedef{PY@tok@c}{\let\PY@it=\textit\def\PY@tc##1{\textcolor[rgb]{0.24,0.48,0.48}{##1}}}
\@namedef{PY@tok@cp}{\def\PY@tc##1{\textcolor[rgb]{0.61,0.40,0.00}{##1}}}
\@namedef{PY@tok@k}{\let\PY@bf=\textbf\def\PY@tc##1{\textcolor[rgb]{0.00,0.50,0.00}{##1}}}
\@namedef{PY@tok@kp}{\def\PY@tc##1{\textcolor[rgb]{0.00,0.50,0.00}{##1}}}
\@namedef{PY@tok@kt}{\def\PY@tc##1{\textcolor[rgb]{0.69,0.00,0.25}{##1}}}
\@namedef{PY@tok@o}{\def\PY@tc##1{\textcolor[rgb]{0.40,0.40,0.40}{##1}}}
\@namedef{PY@tok@ow}{\let\PY@bf=\textbf\def\PY@tc##1{\textcolor[rgb]{0.67,0.13,1.00}{##1}}}
\@namedef{PY@tok@nb}{\def\PY@tc##1{\textcolor[rgb]{0.00,0.50,0.00}{##1}}}
\@namedef{PY@tok@nf}{\def\PY@tc##1{\textcolor[rgb]{0.00,0.00,1.00}{##1}}}
\@namedef{PY@tok@nc}{\let\PY@bf=\textbf\def\PY@tc##1{\textcolor[rgb]{0.00,0.00,1.00}{##1}}}
\@namedef{PY@tok@nn}{\let\PY@bf=\textbf\def\PY@tc##1{\textcolor[rgb]{0.00,0.00,1.00}{##1}}}
\@namedef{PY@tok@ne}{\let\PY@bf=\textbf\def\PY@tc##1{\textcolor[rgb]{0.80,0.25,0.22}{##1}}}
\@namedef{PY@tok@nv}{\def\PY@tc##1{\textcolor[rgb]{0.10,0.09,0.49}{##1}}}
\@namedef{PY@tok@no}{\def\PY@tc##1{\textcolor[rgb]{0.53,0.00,0.00}{##1}}}
\@namedef{PY@tok@nl}{\def\PY@tc##1{\textcolor[rgb]{0.46,0.46,0.00}{##1}}}
\@namedef{PY@tok@ni}{\let\PY@bf=\textbf\def\PY@tc##1{\textcolor[rgb]{0.44,0.44,0.44}{##1}}}
\@namedef{PY@tok@na}{\def\PY@tc##1{\textcolor[rgb]{0.41,0.47,0.13}{##1}}}
\@namedef{PY@tok@nt}{\let\PY@bf=\textbf\def\PY@tc##1{\textcolor[rgb]{0.00,0.50,0.00}{##1}}}
\@namedef{PY@tok@nd}{\def\PY@tc##1{\textcolor[rgb]{0.67,0.13,1.00}{##1}}}
\@namedef{PY@tok@s}{\def\PY@tc##1{\textcolor[rgb]{0.73,0.13,0.13}{##1}}}
\@namedef{PY@tok@sd}{\let\PY@it=\textit\def\PY@tc##1{\textcolor[rgb]{0.73,0.13,0.13}{##1}}}
\@namedef{PY@tok@si}{\let\PY@bf=\textbf\def\PY@tc##1{\textcolor[rgb]{0.64,0.35,0.47}{##1}}}
\@namedef{PY@tok@se}{\let\PY@bf=\textbf\def\PY@tc##1{\textcolor[rgb]{0.67,0.36,0.12}{##1}}}
\@namedef{PY@tok@sr}{\def\PY@tc##1{\textcolor[rgb]{0.64,0.35,0.47}{##1}}}
\@namedef{PY@tok@ss}{\def\PY@tc##1{\textcolor[rgb]{0.10,0.09,0.49}{##1}}}
\@namedef{PY@tok@sx}{\def\PY@tc##1{\textcolor[rgb]{0.00,0.50,0.00}{##1}}}
\@namedef{PY@tok@m}{\def\PY@tc##1{\textcolor[rgb]{0.40,0.40,0.40}{##1}}}
\@namedef{PY@tok@gh}{\let\PY@bf=\textbf\def\PY@tc##1{\textcolor[rgb]{0.00,0.00,0.50}{##1}}}
\@namedef{PY@tok@gu}{\let\PY@bf=\textbf\def\PY@tc##1{\textcolor[rgb]{0.50,0.00,0.50}{##1}}}
\@namedef{PY@tok@gd}{\def\PY@tc##1{\textcolor[rgb]{0.63,0.00,0.00}{##1}}}
\@namedef{PY@tok@gi}{\def\PY@tc##1{\textcolor[rgb]{0.00,0.52,0.00}{##1}}}
\@namedef{PY@tok@gr}{\def\PY@tc##1{\textcolor[rgb]{0.89,0.00,0.00}{##1}}}
\@namedef{PY@tok@ge}{\let\PY@it=\textit}
\@namedef{PY@tok@gs}{\let\PY@bf=\textbf}
\@namedef{PY@tok@gp}{\let\PY@bf=\textbf\def\PY@tc##1{\textcolor[rgb]{0.00,0.00,0.50}{##1}}}
\@namedef{PY@tok@go}{\def\PY@tc##1{\textcolor[rgb]{0.44,0.44,0.44}{##1}}}
\@namedef{PY@tok@gt}{\def\PY@tc##1{\textcolor[rgb]{0.00,0.27,0.87}{##1}}}
\@namedef{PY@tok@err}{\def\PY@bc##1{{\setlength{\fboxsep}{\string -\fboxrule}\fcolorbox[rgb]{1.00,0.00,0.00}{1,1,1}{\strut ##1}}}}
\@namedef{PY@tok@kc}{\let\PY@bf=\textbf\def\PY@tc##1{\textcolor[rgb]{0.00,0.50,0.00}{##1}}}
\@namedef{PY@tok@kd}{\let\PY@bf=\textbf\def\PY@tc##1{\textcolor[rgb]{0.00,0.50,0.00}{##1}}}
\@namedef{PY@tok@kn}{\let\PY@bf=\textbf\def\PY@tc##1{\textcolor[rgb]{0.00,0.50,0.00}{##1}}}
\@namedef{PY@tok@kr}{\let\PY@bf=\textbf\def\PY@tc##1{\textcolor[rgb]{0.00,0.50,0.00}{##1}}}
\@namedef{PY@tok@bp}{\def\PY@tc##1{\textcolor[rgb]{0.00,0.50,0.00}{##1}}}
\@namedef{PY@tok@fm}{\def\PY@tc##1{\textcolor[rgb]{0.00,0.00,1.00}{##1}}}
\@namedef{PY@tok@vc}{\def\PY@tc##1{\textcolor[rgb]{0.10,0.09,0.49}{##1}}}
\@namedef{PY@tok@vg}{\def\PY@tc##1{\textcolor[rgb]{0.10,0.09,0.49}{##1}}}
\@namedef{PY@tok@vi}{\def\PY@tc##1{\textcolor[rgb]{0.10,0.09,0.49}{##1}}}
\@namedef{PY@tok@vm}{\def\PY@tc##1{\textcolor[rgb]{0.10,0.09,0.49}{##1}}}
\@namedef{PY@tok@sa}{\def\PY@tc##1{\textcolor[rgb]{0.73,0.13,0.13}{##1}}}
\@namedef{PY@tok@sb}{\def\PY@tc##1{\textcolor[rgb]{0.73,0.13,0.13}{##1}}}
\@namedef{PY@tok@sc}{\def\PY@tc##1{\textcolor[rgb]{0.73,0.13,0.13}{##1}}}
\@namedef{PY@tok@dl}{\def\PY@tc##1{\textcolor[rgb]{0.73,0.13,0.13}{##1}}}
\@namedef{PY@tok@s2}{\def\PY@tc##1{\textcolor[rgb]{0.73,0.13,0.13}{##1}}}
\@namedef{PY@tok@sh}{\def\PY@tc##1{\textcolor[rgb]{0.73,0.13,0.13}{##1}}}
\@namedef{PY@tok@s1}{\def\PY@tc##1{\textcolor[rgb]{0.73,0.13,0.13}{##1}}}
\@namedef{PY@tok@mb}{\def\PY@tc##1{\textcolor[rgb]{0.40,0.40,0.40}{##1}}}
\@namedef{PY@tok@mf}{\def\PY@tc##1{\textcolor[rgb]{0.40,0.40,0.40}{##1}}}
\@namedef{PY@tok@mh}{\def\PY@tc##1{\textcolor[rgb]{0.40,0.40,0.40}{##1}}}
\@namedef{PY@tok@mi}{\def\PY@tc##1{\textcolor[rgb]{0.40,0.40,0.40}{##1}}}
\@namedef{PY@tok@il}{\def\PY@tc##1{\textcolor[rgb]{0.40,0.40,0.40}{##1}}}
\@namedef{PY@tok@mo}{\def\PY@tc##1{\textcolor[rgb]{0.40,0.40,0.40}{##1}}}
\@namedef{PY@tok@ch}{\let\PY@it=\textit\def\PY@tc##1{\textcolor[rgb]{0.24,0.48,0.48}{##1}}}
\@namedef{PY@tok@cm}{\let\PY@it=\textit\def\PY@tc##1{\textcolor[rgb]{0.24,0.48,0.48}{##1}}}
\@namedef{PY@tok@cpf}{\let\PY@it=\textit\def\PY@tc##1{\textcolor[rgb]{0.24,0.48,0.48}{##1}}}
\@namedef{PY@tok@c1}{\let\PY@it=\textit\def\PY@tc##1{\textcolor[rgb]{0.24,0.48,0.48}{##1}}}
\@namedef{PY@tok@cs}{\let\PY@it=\textit\def\PY@tc##1{\textcolor[rgb]{0.24,0.48,0.48}{##1}}}

\def\PYZbs{\char`\\}
\def\PYZus{\char`\_}
\def\PYZob{\char`\{}
\def\PYZcb{\char`\}}
\def\PYZca{\char`\^}
\def\PYZam{\char`\&}
\def\PYZlt{\char`\<}
\def\PYZgt{\char`\>}
\def\PYZsh{\char`\#}
\def\PYZpc{\char`\%}
\def\PYZdl{\char`\$}
\def\PYZhy{\char`\-}
\def\PYZsq{\char`\'}
\def\PYZdq{\char`\"}
\def\PYZti{\char`\~}
% for compatibility with earlier versions
\def\PYZat{@}
\def\PYZlb{[}
\def\PYZrb{]}
\makeatother


    % For linebreaks inside Verbatim environment from package fancyvrb. 
    \makeatletter
        \newbox\Wrappedcontinuationbox 
        \newbox\Wrappedvisiblespacebox 
        \newcommand*\Wrappedvisiblespace {\textcolor{red}{\textvisiblespace}} 
        \newcommand*\Wrappedcontinuationsymbol {\textcolor{red}{\llap{\tiny$\m@th\hookrightarrow$}}} 
        \newcommand*\Wrappedcontinuationindent {3ex } 
        \newcommand*\Wrappedafterbreak {\kern\Wrappedcontinuationindent\copy\Wrappedcontinuationbox} 
        % Take advantage of the already applied Pygments mark-up to insert 
        % potential linebreaks for TeX processing. 
        %        {, <, #, %, $, ' and ": go to next line. 
        %        _, }, ^, &, >, - and ~: stay at end of broken line. 
        % Use of \textquotesingle for straight quote. 
        \newcommand*\Wrappedbreaksatspecials {% 
            \def\PYGZus{\discretionary{\char`\_}{\Wrappedafterbreak}{\char`\_}}% 
            \def\PYGZob{\discretionary{}{\Wrappedafterbreak\char`\{}{\char`\{}}% 
            \def\PYGZcb{\discretionary{\char`\}}{\Wrappedafterbreak}{\char`\}}}% 
            \def\PYGZca{\discretionary{\char`\^}{\Wrappedafterbreak}{\char`\^}}% 
            \def\PYGZam{\discretionary{\char`\&}{\Wrappedafterbreak}{\char`\&}}% 
            \def\PYGZlt{\discretionary{}{\Wrappedafterbreak\char`\<}{\char`\<}}% 
            \def\PYGZgt{\discretionary{\char`\>}{\Wrappedafterbreak}{\char`\>}}% 
            \def\PYGZsh{\discretionary{}{\Wrappedafterbreak\char`\#}{\char`\#}}% 
            \def\PYGZpc{\discretionary{}{\Wrappedafterbreak\char`\%}{\char`\%}}% 
            \def\PYGZdl{\discretionary{}{\Wrappedafterbreak\char`\$}{\char`\$}}% 
            \def\PYGZhy{\discretionary{\char`\-}{\Wrappedafterbreak}{\char`\-}}% 
            \def\PYGZsq{\discretionary{}{\Wrappedafterbreak\textquotesingle}{\textquotesingle}}% 
            \def\PYGZdq{\discretionary{}{\Wrappedafterbreak\char`\"}{\char`\"}}% 
            \def\PYGZti{\discretionary{\char`\~}{\Wrappedafterbreak}{\char`\~}}% 
        } 
        % Some characters . , ; ? ! / are not pygmentized. 
        % This macro makes them "active" and they will insert potential linebreaks 
        \newcommand*\Wrappedbreaksatpunct {% 
            \lccode`\~`\.\lowercase{\def~}{\discretionary{\hbox{\char`\.}}{\Wrappedafterbreak}{\hbox{\char`\.}}}% 
            \lccode`\~`\,\lowercase{\def~}{\discretionary{\hbox{\char`\,}}{\Wrappedafterbreak}{\hbox{\char`\,}}}% 
            \lccode`\~`\;\lowercase{\def~}{\discretionary{\hbox{\char`\;}}{\Wrappedafterbreak}{\hbox{\char`\;}}}% 
            \lccode`\~`\:\lowercase{\def~}{\discretionary{\hbox{\char`\:}}{\Wrappedafterbreak}{\hbox{\char`\:}}}% 
            \lccode`\~`\?\lowercase{\def~}{\discretionary{\hbox{\char`\?}}{\Wrappedafterbreak}{\hbox{\char`\?}}}% 
            \lccode`\~`\!\lowercase{\def~}{\discretionary{\hbox{\char`\!}}{\Wrappedafterbreak}{\hbox{\char`\!}}}% 
            \lccode`\~`\/\lowercase{\def~}{\discretionary{\hbox{\char`\/}}{\Wrappedafterbreak}{\hbox{\char`\/}}}% 
            \catcode`\.\active
            \catcode`\,\active 
            \catcode`\;\active
            \catcode`\:\active
            \catcode`\?\active
            \catcode`\!\active
            \catcode`\/\active 
            \lccode`\~`\~ 	
        }
    \makeatother

    \let\OriginalVerbatim=\Verbatim
    \makeatletter
    \renewcommand{\Verbatim}[1][1]{%
        %\parskip\z@skip
        \sbox\Wrappedcontinuationbox {\Wrappedcontinuationsymbol}%
        \sbox\Wrappedvisiblespacebox {\FV@SetupFont\Wrappedvisiblespace}%
        \def\FancyVerbFormatLine ##1{\hsize\linewidth
            \vtop{\raggedright\hyphenpenalty\z@\exhyphenpenalty\z@
                \doublehyphendemerits\z@\finalhyphendemerits\z@
                \strut ##1\strut}%
        }%
        % If the linebreak is at a space, the latter will be displayed as visible
        % space at end of first line, and a continuation symbol starts next line.
        % Stretch/shrink are however usually zero for typewriter font.
        \def\FV@Space {%
            \nobreak\hskip\z@ plus\fontdimen3\font minus\fontdimen4\font
            \discretionary{\copy\Wrappedvisiblespacebox}{\Wrappedafterbreak}
            {\kern\fontdimen2\font}%
        }%
        
        % Allow breaks at special characters using \PYG... macros.
        \Wrappedbreaksatspecials
        % Breaks at punctuation characters . , ; ? ! and / need catcode=\active 	
        \OriginalVerbatim[#1,codes*=\Wrappedbreaksatpunct]%
    }
    \makeatother

    % Exact colors from NB
    \definecolor{incolor}{HTML}{303F9F}
    \definecolor{outcolor}{HTML}{D84315}
    \definecolor{cellborder}{HTML}{CFCFCF}
    \definecolor{cellbackground}{HTML}{F7F7F7}
    
    % prompt
    \makeatletter
    \newcommand{\boxspacing}{\kern\kvtcb@left@rule\kern\kvtcb@boxsep}
    \makeatother
    \newcommand{\prompt}[4]{
        \ttfamily\llap{{\color{#2}[#3]:\hspace{3pt}#4}}\vspace{-\baselineskip}
    }
    

    
    % Prevent overflowing lines due to hard-to-break entities
    \sloppy 
    % Setup hyperref package
    \hypersetup{
      breaklinks=true,  % so long urls are correctly broken across lines
      colorlinks=true,
      urlcolor=urlcolor,
      linkcolor=linkcolor,
      citecolor=citecolor,
      }
    % Slightly bigger margins than the latex defaults
    
    \geometry{verbose,tmargin=1in,bmargin=1in,lmargin=1in,rmargin=1in}
    
    

\begin{document}
    
    \maketitle
    
    

    
    \subsection{Python OOPs Concepts
Revision}\label{python-oops-concepts-revision}

    \subsubsection{Python Object-Oriented Programming (OOP) Interview
Preparation}\label{python-object-oriented-programming-oop-interview-preparation}

\paragraph{\texorpdfstring{1. \textbf{Class and
Object:}}{1. Class and Object:}}\label{class-and-object}

\begin{itemize}
\tightlist
\item
  \textbf{Definition:}

  \begin{itemize}
  \tightlist
  \item
    A class is a blueprint for creating objects. It defines attributes
    (data) and methods (functions) that the objects will have.
  \item
    An object is an instance of a class.
  \end{itemize}
\item
  \textbf{Example:}

  \begin{itemize}
  \tightlist
  \item
    Class: \texttt{Car}
  \item
    Object: \texttt{my\_car\ =\ Car()}
  \end{itemize}
\end{itemize}

\paragraph{\texorpdfstring{2. \textbf{Attributes and
Methods:}}{2. Attributes and Methods:}}\label{attributes-and-methods}

\begin{itemize}
\item
  \textbf{Attributes:}

  \begin{itemize}
  \tightlist
  \item
    Characteristics or properties of an object.
  \end{itemize}
\item
  \textbf{Methods:}

  \begin{itemize}
  \tightlist
  \item
    Functions associated with a class.
  \end{itemize}
\item
  \textbf{Example:}

\begin{Shaded}
\begin{Highlighting}[]
\KeywordTok{class}\NormalTok{ Dog:}
    \KeywordTok{def} \FunctionTok{__init__}\NormalTok{(}\VariableTok{self}\NormalTok{, name, age):}
        \VariableTok{self}\NormalTok{.name }\OperatorTok{=}\NormalTok{ name}
        \VariableTok{self}\NormalTok{.age }\OperatorTok{=}\NormalTok{ age}

    \KeywordTok{def}\NormalTok{ bark(}\VariableTok{self}\NormalTok{):}
        \ControlFlowTok{return} \StringTok{"Woof!"}

\NormalTok{my_dog }\OperatorTok{=}\NormalTok{ Dog(}\StringTok{"Buddy"}\NormalTok{, }\DecValTok{5}\NormalTok{)}
\BuiltInTok{print}\NormalTok{(my_dog.name)  }\CommentTok{# Output: Buddy}
\BuiltInTok{print}\NormalTok{(my_dog.bark())  }\CommentTok{# Output: Woof!}
\end{Highlighting}
\end{Shaded}
\end{itemize}

\paragraph{\texorpdfstring{3. \textbf{Constructor and
Destructor:}}{3. Constructor and Destructor:}}\label{constructor-and-destructor}

\begin{itemize}
\item
  \textbf{Constructor (\texttt{\_\_init\_\_}):}

  \begin{itemize}
  \tightlist
  \item
    Special method invoked when an object is instantiated.
  \item
    Used for initializing object attributes.
  \end{itemize}
\item
  \textbf{Destructor (\texttt{\_\_del\_\_}):}

  \begin{itemize}
  \tightlist
  \item
    Special method invoked when an object is about to be destroyed.
  \item
    Used for cleanup activities.
  \end{itemize}
\item
  \textbf{Example:}

\begin{Shaded}
\begin{Highlighting}[]
\KeywordTok{class}\NormalTok{ MyClass:}
    \KeywordTok{def} \FunctionTok{__init__}\NormalTok{(}\VariableTok{self}\NormalTok{):}
        \BuiltInTok{print}\NormalTok{(}\StringTok{"Constructor called."}\NormalTok{)}

    \KeywordTok{def} \FunctionTok{__del__}\NormalTok{(}\VariableTok{self}\NormalTok{):}
        \BuiltInTok{print}\NormalTok{(}\StringTok{"Destructor called."}\NormalTok{)}

\NormalTok{obj }\OperatorTok{=}\NormalTok{ MyClass()  }\CommentTok{# Output: Constructor called.}
\KeywordTok{del}\NormalTok{ obj  }\CommentTok{# Output: Destructor called.}
\end{Highlighting}
\end{Shaded}
\end{itemize}

\paragraph{\texorpdfstring{4.
\textbf{Inheritance:}}{4. Inheritance:}}\label{inheritance}

\begin{itemize}
\item
  \textbf{Definition:}

  \begin{itemize}
  \tightlist
  \item
    The mechanism of creating a new class from an existing class.
  \item
    Allows a class to inherit attributes and methods from another class.
  \end{itemize}
\item
  \textbf{Example:}

\begin{Shaded}
\begin{Highlighting}[]
\KeywordTok{class}\NormalTok{ Animal:}
    \KeywordTok{def}\NormalTok{ sound(}\VariableTok{self}\NormalTok{):}
        \ControlFlowTok{return} \StringTok{"Sound"}

\KeywordTok{class}\NormalTok{ Dog(Animal):}
    \KeywordTok{def}\NormalTok{ sound(}\VariableTok{self}\NormalTok{):}
        \ControlFlowTok{return} \StringTok{"Woof!"}

\NormalTok{my_dog }\OperatorTok{=}\NormalTok{ Dog()}
\BuiltInTok{print}\NormalTok{(my_dog.sound())  }\CommentTok{# Output: Woof!}
\end{Highlighting}
\end{Shaded}
\end{itemize}

\paragraph{\texorpdfstring{5.
\textbf{Polymorphism:}}{5. Polymorphism:}}\label{polymorphism}

\begin{itemize}
\item
  \textbf{Definition:}

  \begin{itemize}
  \tightlist
  \item
    The ability of objects to take on multiple forms.
  \item
    Allows methods to behave differently based on the object calling
    them.
  \end{itemize}
\item
  \textbf{Example:}

\begin{Shaded}
\begin{Highlighting}[]
\KeywordTok{class}\NormalTok{ Animal:}
    \KeywordTok{def}\NormalTok{ sound(}\VariableTok{self}\NormalTok{):}
        \ControlFlowTok{return} \StringTok{"Sound"}

\KeywordTok{class}\NormalTok{ Dog(Animal):}
    \KeywordTok{def}\NormalTok{ sound(}\VariableTok{self}\NormalTok{):}
        \ControlFlowTok{return} \StringTok{"Woof!"}

\KeywordTok{class}\NormalTok{ Cat(Animal):}
    \KeywordTok{def}\NormalTok{ sound(}\VariableTok{self}\NormalTok{):}
        \ControlFlowTok{return} \StringTok{"Meow!"}

\KeywordTok{def}\NormalTok{ make_sound(animal):}
    \ControlFlowTok{return}\NormalTok{ animal.sound()}

\NormalTok{my_dog }\OperatorTok{=}\NormalTok{ Dog()}
\NormalTok{my_cat }\OperatorTok{=}\NormalTok{ Cat()}
\BuiltInTok{print}\NormalTok{(make_sound(my_dog))  }\CommentTok{# Output: Woof!}
\BuiltInTok{print}\NormalTok{(make_sound(my_cat))  }\CommentTok{# Output: Meow!}
\end{Highlighting}
\end{Shaded}
\end{itemize}

\paragraph{\texorpdfstring{6.
\textbf{Encapsulation:}}{6. Encapsulation:}}\label{encapsulation}

\begin{itemize}
\item
  \textbf{Definition:}

  \begin{itemize}
  \tightlist
  \item
    Bundling of data (attributes) and methods that operate on the data
    within a single unit (class).
  \item
    Data hiding mechanism to restrict access to certain attributes or
    methods.
  \end{itemize}
\item
  \textbf{Example:}

\begin{Shaded}
\begin{Highlighting}[]
\KeywordTok{class}\NormalTok{ BankAccount:}
    \KeywordTok{def} \FunctionTok{__init__}\NormalTok{(}\VariableTok{self}\NormalTok{):}
        \VariableTok{self}\NormalTok{.balance }\OperatorTok{=} \DecValTok{0}

    \KeywordTok{def}\NormalTok{ deposit(}\VariableTok{self}\NormalTok{, amount):}
        \VariableTok{self}\NormalTok{.balance }\OperatorTok{+=}\NormalTok{ amount}

    \KeywordTok{def}\NormalTok{ withdraw(}\VariableTok{self}\NormalTok{, amount):}
        \ControlFlowTok{if} \VariableTok{self}\NormalTok{.balance }\OperatorTok{>=}\NormalTok{ amount:}
            \VariableTok{self}\NormalTok{.balance }\OperatorTok{-=}\NormalTok{ amount}
        \ControlFlowTok{else}\NormalTok{:}
            \BuiltInTok{print}\NormalTok{(}\StringTok{"Insufficient funds."}\NormalTok{)}

\NormalTok{account }\OperatorTok{=}\NormalTok{ BankAccount()}
\NormalTok{account.deposit(}\DecValTok{1000}\NormalTok{)}
\NormalTok{account.withdraw(}\DecValTok{500}\NormalTok{)}
\BuiltInTok{print}\NormalTok{(account.balance)  }\CommentTok{# Output: 500}
\end{Highlighting}
\end{Shaded}
\end{itemize}

\paragraph{\texorpdfstring{7.
\textbf{Abstraction:}}{7. Abstraction:}}\label{abstraction}

\begin{itemize}
\item
  \textbf{Definition:}

  \begin{itemize}
  \tightlist
  \item
    Process of hiding the complex implementation details and showing
    only the necessary features of an object.
  \end{itemize}
\item
  \textbf{Example:}

\begin{Shaded}
\begin{Highlighting}[]
\ImportTok{from}\NormalTok{ abc }\ImportTok{import}\NormalTok{ ABC, abstractmethod}

\KeywordTok{class}\NormalTok{ Animal(ABC):}
    \AttributeTok{@abstractmethod}
    \KeywordTok{def}\NormalTok{ sound(}\VariableTok{self}\NormalTok{):}
        \ControlFlowTok{pass}

\KeywordTok{class}\NormalTok{ Dog(Animal):}
    \KeywordTok{def}\NormalTok{ sound(}\VariableTok{self}\NormalTok{):}
        \ControlFlowTok{return} \StringTok{"Woof!"}

\NormalTok{my_dog }\OperatorTok{=}\NormalTok{ Dog()}
\BuiltInTok{print}\NormalTok{(my_dog.sound())  }\CommentTok{# Output: Woof!}
\end{Highlighting}
\end{Shaded}
\end{itemize}

\paragraph{\texorpdfstring{8. \textbf{Getter and Setter
Methods:}}{8. Getter and Setter Methods:}}\label{getter-and-setter-methods}

\begin{itemize}
\item
  \textbf{Definition:}

  \begin{itemize}
  \tightlist
  \item
    Getter methods are used to access the value of a private attribute.
  \item
    Setter methods are used to modify the value of a private attribute.
  \end{itemize}
\item
  \textbf{Example:}

\begin{Shaded}
\begin{Highlighting}[]
\KeywordTok{class}\NormalTok{ Person:}
    \KeywordTok{def} \FunctionTok{__init__}\NormalTok{(}\VariableTok{self}\NormalTok{, name):}
        \VariableTok{self}\NormalTok{.__name }\OperatorTok{=}\NormalTok{ name}

    \KeywordTok{def}\NormalTok{ get_name(}\VariableTok{self}\NormalTok{):}
        \ControlFlowTok{return} \VariableTok{self}\NormalTok{.__name}

    \KeywordTok{def}\NormalTok{ set_name(}\VariableTok{self}\NormalTok{, name):}
        \VariableTok{self}\NormalTok{.__name }\OperatorTok{=}\NormalTok{ name}

\NormalTok{person }\OperatorTok{=}\NormalTok{ Person(}\StringTok{"Alice"}\NormalTok{)}
\BuiltInTok{print}\NormalTok{(person.get_name())  }\CommentTok{# Output: Alice}
\NormalTok{person.set_name(}\StringTok{"Bob"}\NormalTok{)}
\BuiltInTok{print}\NormalTok{(person.get_name())  }\CommentTok{# Output: Bob}
\end{Highlighting}
\end{Shaded}
\end{itemize}

\paragraph{\texorpdfstring{9. \textbf{Class and Instance
Variables:}}{9. Class and Instance Variables:}}\label{class-and-instance-variables}

\begin{itemize}
\item
  \textbf{Class Variables:}

  \begin{itemize}
  \tightlist
  \item
    Shared among all instances of a class.
  \item
    Defined within the class but outside any methods.
  \end{itemize}
\item
  \textbf{Instance Variables:}

  \begin{itemize}
  \tightlist
  \item
    Unique to each instance of a class.
  \item
    Defined inside the constructor using \texttt{self}.
  \end{itemize}
\item
  \textbf{Example:}

\begin{Shaded}
\begin{Highlighting}[]
\KeywordTok{class}\NormalTok{ Circle:}
\NormalTok{    pi }\OperatorTok{=} \FloatTok{3.14}  \CommentTok{# Class variable}

    \KeywordTok{def} \FunctionTok{__init__}\NormalTok{(}\VariableTok{self}\NormalTok{, radius):}
        \VariableTok{self}\NormalTok{.radius }\OperatorTok{=}\NormalTok{ radius  }\CommentTok{# Instance variable}

\NormalTok{circle1 }\OperatorTok{=}\NormalTok{ Circle(}\DecValTok{5}\NormalTok{)}
\NormalTok{circle2 }\OperatorTok{=}\NormalTok{ Circle(}\DecValTok{10}\NormalTok{)}
\BuiltInTok{print}\NormalTok{(circle1.radius)  }\CommentTok{# Output: 5}
\BuiltInTok{print}\NormalTok{(circle2.radius)  }\CommentTok{# Output: 10}
\BuiltInTok{print}\NormalTok{(Circle.pi)  }\CommentTok{# Output: 3.14}
\end{Highlighting}
\end{Shaded}
\end{itemize}

\paragraph{\texorpdfstring{10. \textbf{Method
Overriding:}}{10. Method Overriding:}}\label{method-overriding}

\begin{itemize}
\item
  \textbf{Definition:}

  \begin{itemize}
  \tightlist
  \item
    Providing a new implementation for a method in the subclass, which
    is already present in the parent class.
  \end{itemize}
\item
  \textbf{Example:}

\begin{Shaded}
\begin{Highlighting}[]
\KeywordTok{class}\NormalTok{ Animal:}
    \KeywordTok{def}\NormalTok{ sound(}\VariableTok{self}\NormalTok{):}
        \ControlFlowTok{return} \StringTok{"Sound"}

\KeywordTok{class}\NormalTok{ Dog(Animal):}
    \KeywordTok{def}\NormalTok{ sound(}\VariableTok{self}\NormalTok{):}
        \ControlFlowTok{return} \StringTok{"Woof!"}

\NormalTok{my_dog }\OperatorTok{=}\NormalTok{ Dog()}
\BuiltInTok{print}\NormalTok{(my_dog.sound())  }\CommentTok{# Output: Woof!}
\end{Highlighting}
\end{Shaded}
\end{itemize}

\begin{center}\rule{0.5\linewidth}{\linethickness}\end{center}

    \begin{tcolorbox}[breakable, size=fbox, boxrule=1pt, pad at break*=1mm,colback=cellbackground, colframe=cellborder]
\prompt{In}{incolor}{ }{\boxspacing}
\begin{Verbatim}[commandchars=\\\{\}]

\end{Verbatim}
\end{tcolorbox}


    % Add a bibliography block to the postdoc
    
    
    
\end{document}
